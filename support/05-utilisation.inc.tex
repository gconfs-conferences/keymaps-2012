\section{Utilisation}



\subsection{Bon usage d’un clavier}

\begin{frame}{Bon usage d’un clavier}
    Objectifs~:
    \begin{itemize}
        \item Saisir rapidement et avec précision \pause

        \item Avoir une posture confortable et saine
    \end{itemize}

    Moyens~:
    \begin{itemize}
        \item Bien positionner son clavier par rapport à soi \pause

        \item Bien positionner ses doigts sur les touches au repos \pause

        \item Pour chaque touche, utiliser le doigt associé \pause

        \item Ne pas jongler entre écran et clavier (donc ne pas regarder le clavier)
    \end{itemize}
\end{frame}



\subsection{Apprentissage}

\begin{frame}{Les utilitaires}
    \begin{itemize}
        \item But~: parvenir à maîtriser une keymap ou un clavier le plus
          rapidement-efficacement possible \pause
        \item Demande un peu d’exercice chaque jours pendant une ou deux
          semaines, un peu de pratique, et c’est gagné~! \pause

	\item GNU Typist \pause
        \item Klavaro (Windows, GNU/Linux, toute keymap) \pause
        \item Ktouch (GNU/Linux) \pause
        \item TypeTrainer4Mac (Mac OS X) \pause
	\item Le pld Bépo (interactif)
    \end{itemize}
\end{frame}

\begin{frame}{Conditions}
    Pour l’épitéen typique~:
    \begin{itemize}
        \item Demande un minimum de motivation (penser aux avantages) \pause

        \item \emph{À éviter pendant un rush~!!!} \pause

        \item Les vacances sont le moment idéal (pas besoin de productivité)
    \end{itemize}
\end{frame}



\subsection{Changer la keymap d’un clavier}

\begin{frame}{Microsoft Windows}
\end{frame}

\begin{frame}{Mac OS X}
\end{frame}

\begin{frame}{GNU/Linux et *BSD}
\end{frame}
