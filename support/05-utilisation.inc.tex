\section{Utilisation}



\subsection{Bon usage d’un clavier}

\begin{frame}{Bon usage d’un clavier}
    Objectifs~:
    \begin{itemize}
        \item Saisir rapidement et avec précision \pause

        \item Avoir une posture confortable et saine
    \end{itemize}
    \pause

    Moyens~:
    \begin{itemize}
        \item Bien positionner son clavier par rapport à soi \pause

        \item Bien positionner ses doigts sur les touches au repos \pause

        \item Pour chaque touche, utiliser le doigt associé \pause

        \item Ne pas jongler entre écran et clavier (donc ne pas regarder le clavier)
    \end{itemize}
\end{frame}



\subsection{Apprentissage}

\begin{frame}{Les utilitaires}
    \begin{itemize}
        \item But~: parvenir à maîtriser une keymap ou un clavier le plus
          rapidement-efficacement possible \pause
        \item Demande un peu d’exercice chaque jours pendant une ou deux
          semaines, un peu de pratique, et c’est gagné~! \pause

	\item GNU Typist \pause
        \item Klavaro (Windows, GNU/Linux, toute keymap) \pause
        \item Ktouch (GNU/Linux) \pause
        \item TypeTrainer4Mac (Mac OS X) \pause
	\item Le pld Bépo (interactif)
    \end{itemize}
\end{frame}

\begin{frame}{Conditions}
    Pour l’épitéen typique~:
    \begin{itemize}
        \item Demande un minimum de motivation (penser aux avantages) \pause

        \item \emph{À éviter pendant un rush~!!!} \pause

        \item Les vacances sont le moment idéal (pas besoin de productivité)
    \end{itemize}
\end{frame}



\subsection{Changer la keymap d’un clavier}

\begin{frame}{Microsoft Windows}
    Deux possibilités (reviennent à peu près au même)~:
    \pause
    \begin{itemize}
        \item Dans le panneau de configuration, ouvrir l’item «~Clavier~» ou
          «~Service de texte et de langues~», puis ajouter la keymap voulue et
          la mettre par défaut dans l’onglet «~Général~». \pause

        \item Ajouter la keymap voulue dans le menu de la «~Barre de langue~»
          et la sélectionner avant de l’utiliser.
    \end{itemize}
\end{frame}

\begin{frame}{Mac OS X}
    \begin{itemize}
        \item Tout se passe dans le menu «~International~» (dans les
          préférences système)~: item «~Langue et texte~», onglet «~Input~»
          \pause

        \item Il s’agit d’activer des keymaps, pour ensuite les sélectionner
          dans l’icône de la barre de menus.
    \end{itemize}
\end{frame}

\begin{frame}{GNU/Linux et *BSD}
    \begin{itemize}
        \item Pour changer la keymap de X~: \pause
            \begin{itemize}
                \item Par interface graphique~: dépend du gestionnaire de
                  bureau utilisé \pause

                \item En ligne de commande~:
                  \newline \texttt{setxkbmap us}
                  \newline \texttt{setxkbmap us intl}
                  \newline \texttt{setxkbmap ca multi}
                  \newline \texttt{setxkbmap fr}
                  \newline \texttt{setxkbmap fr bepo}
                  ...
            \end{itemize}
            \pause

        \item Pour changer la keymap des tty~: dépend fortement du système et
          de la distribution utilisés…\pause
    \end{itemize}
\end{frame}
